\documentclass[hidelinks,12pt,xcolor=table]{article}
\usepackage[a4paper, margin=0.7in]{geometry}
\usepackage[utf8]{inputenc}
\usepackage{multicol}
\usepackage{authblk}
\usepackage{graphicx}
\usepackage{amsmath}
\usepackage[colorinlistoftodos]{todonotes}
\usepackage[colorlinks=true, allcolors=blue]{hyperref}
\usepackage{enumitem}
\newenvironment{Figure}
  {\par\medskip\noindent\minipage{\linewidth}}
  {\endminipage\par\medskip}
\usepackage{lipsum}
\usepackage{hyperref}

\title{Lung cancer detection using CXRs and pattern recognition
\date{}
}

\author[1]{Author A}
\author[1]{Author B}
\author[1]{Author C}
\author[1]{Author D}
\author[1]{Author E}
\affil[1]{Department of Computer Science, \LaTeX\ University}
\renewcommand\Authands{ and }
\begin{document}
\maketitle
\begin{abstract}
Cancer is a leading cause of death worldwide.
Lung cancer is a type of cancer that is considered as one of the most leading causes of death globally.
This paper aims to elaborate difference between 4 strategies based on 3 different classifiers 
(SVM  \& KNN \& XGBOOST) and Wavelet transformation by it's all families as a Feature Extraction.
and Wavelet transformation by it’s all families as a Feature Extraction.
\end{abstract}

\begin{multicols}{2}
\section{Introduction}
Lung cancer is one of the most serious cancers in the world. Survival from lung cancer is
directly related to its growth at its detection. The earlier the detection is, the higher the
chances of successful treatment are. Chest X-ray image has been used for detecting lung
cancer for a long time. The early detection and diagnosis of pulmonary nodules in chest
X-ray image are among the most challenging clinical tasks performed by radiologists.

Many techniques have been developed in order to
increase the detection accuracy rates of lung cancer in CAD systems.
This comparative study will elaborate difference between 4 strategies based on 3 different classifiers (SVM  \& KNN \& XGBOOST) and Wavelet transformation by it's all families as a Feature Extraction.

\begin{enumerate}
\item classify CXRs without lung extraction
\item classify CXRs with lung extraction
\item Nodule Detection in CXRs without lung extraction
\item Nodule Detection in CXRs with lung extraction
\end{enumerate}
\section{Related Work}
Many techniques have been developed in order to
increase the detection accuracy rates of lung cancer in CAD
systems. For instance, Hardie et al. [1] presented a CAD
system for pulmonary nodule detection in chest radiography.
The proposed system was tested using a data set that
consists of 167 chest radiography that contains 181 lung
nodules; the system utilized adaptive distance-based
threshold algorithm for nodule segmentation, after that,
features were computed for each nodule using geometric
features, intensity features and gradient features. Lastly, a
Fisher linear discriminant classifier was used to c1assity the
computed features. The system could detect 78.1 \% of those
nodules. Lee et al. [2] proposed a lung nodule detection
using an ensemble classifier aided by clustering. Lung scans
of 32 patients who included 5721 images were used to test
the method. The system obtained sensitivity of 98.33\% and
specificity of 97.11 \%.An automatic method for lung cancer detection was
also introduced by Sousa et al. [3]. The system has six
stages, namely, thorax extraction, lung extraction, lung
reconstruction, structure's extraction; tubular structures
elimination, and false-positive reduction. Each stage
performs a specific task that resulted in detecting lung
nodules. The sensitivity achieved was 84.84\%, and
specificity achieved was 96.15\%.
These were examples of systems that have been developed
to detect lung cancer. General issues of existing systems are
the high number of false positives and false negatives.
Therefore, there is a continuous need and importance to
develop computer-aided diagnosis to help in lung cancer
detection and prediction.
According To resources mentioned in references The most technique used to extract lung field is Active shape model (ASM), and the most classifier used is KNN and SVM, So this work aims to do comparative study based on three different classifier KNN, SVM , XGBoost.  
\section{Materials}
The proposed systems was tested using JSRT Database includes 154 abnormal chest radiographs, each with a solitary pulmonary nodule, and 93 non-nodule chest radiographs. These original screen-film images were digitized with a 0.175-mm pixel size, matrix size of 2048 2048, and 12 bits of gray scale.
\section{Method}
\subsection{Feature Extraction}
Feature Extraction based on Wavelet Transform
Feature extraction is an important step in CAD systems.
The extraction of features that represent medical images of
specific organs is an important issue. Multi-resolution
methods such as Wavelet Transform (WT) have been widely
used in feature extraction due to its quality compared with
other feature extraction methods.
In PYHTON, wavelet is implemented as a library called pywavelets.its
offers many wavelet functions that can be utilized. Wavelet
families such as Haar (haar),
Daubechies (db),
Symlets (sym),
Coiflets (coif),
Biorthogonal (bior),
Reverse biorthogonal (rbio),
“Discrete” FIR approximation of Meyer wavelet (dmey),
Gaussian wavelets (gaus),
Mexican hat wavelet (mexh),
Morlet wavelet (morl),
Complex Gaussian wavelets (cgau),
Shannon wavelets (shan),
Frequency B-Spline wavelets (fbsp),
Complex Morlet wavelets (cmor),
with scales(Levels)  from 1 to 6 level,
with (ad,da,dd) horizental,vertcal,diagonal detailes coefficients.
such as family(haar) , level(4)  that return (ad,da,dd) detailes coefficients and one of them is classified 


\subsection{Feature Selection}
Once the feature extraction stage has been executed, a
huge number of coefficients will be produced. It is
important to reduce the coefficients by selecting those
coefficients that contains the most important information
that would contribute to high accuracy and ignoring the
remaining. For this reason,we use (low variance) to remove these coefficients\\
VarianceThreshold is a simple baseline approach to feature selection. It removes all features whose variance doesn’t meet some threshold. By default, it removes all zero-variance features, i.e. features that have the same value in all samples.


\subsection{Classification}
python offers Simple and efficient tools for data mining and data analysis
Accessible to everybody, and reusable in various contexts
Built on NumPy, SciPy, and matplotlib 
Open source, commercially usable - BSD license called scikit-learn\\
Classification methods used in proposed  systems are support vector machine(SVM),K-Nearest Neighbor (KNN)
,xgboost classifier.\\
dataset are dived to 80\% training set and 20\% testing set then we shuffle data and repeat that
process of classification 50 times and get average result


\section{Experimental Study}

\subsection{classify CXRs without lung extraction (system 1)}
\subsubsection{Feature Extraction}
Feature Extraction based on Wavelet Transform with families
'haar', 'db', 'sym', 'coif', 'bior', 'rbio', 'dmey', 'gaus',
'mexh', 'morl', 'cgau', 'shan', 'fbsp', 'cmor',
and scales(Levels) 4,5,6   



\subsubsection{Classification}

We dived the dataset to 80\% training set and 20\% testing set then we shuffle data and repeat that process 50 times and get average result we do that with 3 different classifier KNN, SVM and XGBoost

\subsubsection{ROC}

\begin{center}
  \centering
  \includegraphics[width=\linewidth]{ROCs_FROCs/sys1.png}
\end{center}



% \clearpage


\subsection{classify CXRs with lung extraction (system 2)}
\subsubsection{Preprocessing}
To eliminate false positives that might occur outside of
the lung, it is important to obtain a very accurate
segmentation of the lung boundaries. In this work we used ASM(active shape model) machine learning technique with dataset lung field landmarks to extract lung from image


\subsubsection{Feature Extraction}
Feature Extraction based on Wavelet Transform with families
'haar', 'db', 'sym', 'coif', 'bior', 'rbio', 'dmey', 'gaus',
'mexh', 'morl', 'cgau', 'shan', 'fbsp', 'cmor',
and scales(Levels) 4,5,6  

\subsubsection{Classification}
We dived the dataset to 80\% training set and 20\% testing set then we shuffle data and repeat that process 50 times and get average result we do that with 3 different classifier KNN, SVM and XGBoost

\subsubsection{ROC}
\begin{center}
  \centering
  \includegraphics[width=\linewidth]{ROCs_FROCs/sys2.png}
  \label{fig:sys2}
\end{center}
%\clearpage



\subsection{Nodule Detection in CXRs without lung extraction(system 3)}
\subsubsection{Preprocessing}
In this expirement we divide our images to patches each patch equal to 64*64 pixel\\
The abnormal regions were selected based on the
coordinates that are provided with the dataset, and the
normal regions were extracted randomly from the 93 non nodule images
for each abnormal region we get 7 different orientation\\

1- orignal postion

2- 90 degree rotation

3- 180 degree rotation

4- 270 degree rotation

5- flip horizontal

6- flip horizental with 90 degree rotation

6- flip horizontal with 180 degree rotation

7- flip horizontal with 270 degree rotation\\

In JSRT there exist 156 noduled image
from each image we get 7 different patches,
so we will have 1078 noduled patches\\

Then we divide normal images into equal patches (64*64 pixel)
In JSRT there exist 93 normal image, 
each image is divided into 1024 patch
so we will have 95,232 normal patch

\subsubsection{Feature Extraction}
Feature Extraction based on Wavelet Transform with families
'haar', 'db', 'sym', 'coif', 'bior', 'rbio', 'dmey', 'gaus',
'mexh', 'morl', 'cgau', 'shan', 'fbsp', 'cmor',
and scales(Levels) 1,2,3 

\subsubsection{Classification}
At first we choose 1078 random sample from normal patches and take all noduled patches 
then we divide them to 80\% training set and 20\% testing set then we shuffle data and repeat that process 50 times and get average result we do that with 3 different classifier KNN, SVM and XGBoost
%\clearpage

\subsubsection{FROC}
\begin{center}
  \centering
  \includegraphics[width=\linewidth]{ROCs_FROCs/sys3.png}
  \label{fig:sys2}
\end{center}

%\clearpage


\subsection{Nodule Detection in CXRs with lung extraction (system 4)}
\subsubsection{Preprocessing}
 We use Active Shape Model to extract lung field from CXRs and divide our images to patches each patch equal to 64*64 pixel\\


n this expirement we divide our images to patches each patch equal to 64*64 pixel\\
The abnormal regions were selected based on the
coordinates that are provided with the dataset, and the
normal regions were extracted randomly from the 93 non nodule images
for each abnormal region we get 7 different orientation\\

1- orignal postion

2- 90 degree rotation

3- 180 degree rotation

4- 270 degree rotation

5- flip horizontal

6- flip horizental with 90 degree rotation

6- flip horizontal with 180 degree rotation

7- flip horizontal with 270 degree rotation\\

In JSRT there exist 156 noduled image
from each image we get 7 different patches
so we will have 1078 noduled patches\\

Then we divide normal images into equal patches (64*64 pixel)
In JSRT there exist 93 normal image, 
each image is divided into 1024 patch and remove patches that are outside the lung

\subsubsection{Feature Extraction}
Feature Extraction based on Wavelet Transform with families
'haar', 'db', 'sym', 'coif', 'bior', 'rbio', 'dmey', 'gaus',
'mexh', 'morl', 'cgau', 'shan', 'fbsp', 'cmor',
and scales(Levels) 1,2,3 

\subsubsection{Classification}
At first we choose 1078 random sample from normal patches and take all noduled patches 
then we divide them to 80\% training set and 20\% testing set then we shuffle data and repeat that process 50 times and get average result we do that with 3 different classifier KNN, SVM and XGBoost


%\newpage

\subsubsection{FROC}
\begin{center}
  \centering
  \includegraphics[width=\linewidth]{ROCs_FROCs/sys4-bior39-ad-lvl1.png}
  \label{fig:sys1-db23-ad-lvl5}
\end{center}

\section{Result}
The table shown in best results achieved in following table :\\

\begin{center}
\centering
%\caption{Results}
\label{my-label}
\begin{tabular}{|c|c|c|c|}
\hline & SVM & KNN & XGBOOST                        \\ \hline
System 1 & 74.44\% & 84.22\% & {\color[HTML]{FE0000} 96.0\%}  \\ \hline
System 2 & 83.0\%  & {\color[HTML]{FE0000} 83.8\%}  & 82.6\%                         \\ \hline
System 3 & 82.25\% & {\color[HTML]{FE0000} 93.14\%} & 92.13\%                        \\ \hline
System 4 & 90.22\% & 86.34\% & {\color[HTML]{FE0000} 99.98\%} \\ \hline
\end{tabular}
\end{center}

 Best result achieved in system 1  with SVM used db32 level 5 (ad) as feature extraction
and with KNN used coif4 level 5 (ad) as feature extraction
and with XGBOOST used bior5.5 level 4 (ad) as feature extraction\\

 Best result achieved in  system 2 with SVM used sym6 level 4(ad) as feature extraction
and with KNN used sym6 level 4(ad) as feature extraction
and with XGBOOST used sym6 level 4(ad) as feature extraction\\

 Best result achieved in  system 3 with SVM , bior3.9 used level 1 (ad) as feature extraction
and with KNN , bior2.2 used level 2 (ad) as feature extraction
and with XGBOOST used db3  level 3 (ad) as feature extraction\\

 Best result achieved in  system 4 with SVM used db7 , level 2 (ad) as feature extraction
and with KNN used rbio3.1  level 3 (ad) as feature extraction
and with XGBOOSt used sym4  level 2 (dd) as feature extraction\\

\section{Conclusion}
After trying different expirements we can conclude that:\\
\begin{itemize}
\item System with lung field extraction and segmented patches get the best accuracy compared with others systems\\
\item XGboost has better results with wavelet Transformation\\
\end{itemize}

\section{References}
\begin{enumerate}
\item Russell C. Hardie, Steven K. Rogers, Terry Wilson, Adam Rogers;
Performance analysis of a new computer aided detection system for
identiJYing lung nodules on chest radiographs. Medical Image
Analysis 12 (200S); pp. 240-25S.

\item S.L.A. Lee, A.Z. Kouzani, EJ. Hu, Random forest based lung nodule
classification aided by clustering. Computerized Medical Imaging and
Graphics 34 (20 I 0); pp. 535-542

\item Joao Rodrigo Ferreira da Silva Sousa, Arist6fanes CorrVea Silva,
Anselmo Cardoso de Paiva, Rodolfo Acatauassu Nunes, Methodology
for automatic detection of lung nodules in computerized tomography
images. Computer Methods and Programs in Biomedicine 9S (2010);
pp. I-14.

% \item Multi-scale Nodule Detection in Chest Radiographs

% Arnold M.R. Schilham, Bram van Ginneken, and Marco Loog
% Image Sciences Institute, University Medical Center Utrecht, The Netherlands\label{1}

% \item Optimal image feature set for detecting lung nodules on chest X-ray images 

% Jun Wei, Yoshihiro Hagihara, Akinobu Shimizu, Hidefumi Kobatake\label{2}


% \item Detection of Lung Nodule Candidates in Chest Radiographs

% Carlos S. Pereira, Hugo Fernandes1, Ana Maria Mendonwca,
% and Aurmelio Campilho\label{3}


% \item Lung Cancer Classification Using Image Processing 

% International Journal of Engineering and Innovative Technology (IJEIT) Volume 2, Issue 3, September 2012 \label{4}


% \item Computer Aided Diagnosis System based on Machine Learning Techniques for Lung Cancer \label{5}

% Hamada R. H. AI-Absi, Brahim Belhaouari Samir, Khaled Bashir Shaban, and Suziah Sulaiman 

% \item A Computer Aided Pulmonary Nodule Detection System Using Multiple Massive Training SVMs\label{6}

% Zhenghao Shi1, Minghua Zhao1, Lifeng He4, Yinghui Wang1, Ming Zhang2 and Kenji Suzuki3

% \item A Computer Aided Diagnosis System for Lung Cancer based on Statistical and Machine Learning Techniques \label{7}

% Hamada R. H. Al-Absi1, Brahim Belhaouari Samir2*, Suziah Sulaiman1

% \item A computer-aided diagnosis system for detection of lung nodules in chest radiographs with an evaluation on a public database\label{8}

% Arnold M.R. Schilham *, Bram van Ginneken, Marco Loog

% \item Classification of Chest Lesions with Using Fuzzy  C-Means Algorithm and Support Vector Machines \label{10}

% Donia Ben Hassen1, Hassen Taleb1, Ismahen Ben Yaacoub2, and Najla Mnif2


% \item Geometrical and texture features estimation of lung cancer and TB images using chest X-ray database \label{11}

% S.A. Patil

% \item Computerized Detection of Lung Nodules by Means of virtual Dual energy Radiography \label{12}

% Sheng Chen* and Kenji Suzuki, Member, IEEE 

% \item On the Combination of Wavelet and Curvelet for Feature Extraction to Classify Lung Cancer on Chest Radiographs \label{13}

% Hamada R. H. Al-Absi, Brahim Belhaouari Samir, Taha Alhersh and Suziah Sulaiman

% \item Computer-aided Diagnosis for the Detection and Classifcation of Lung Cancers on Chest Radiographs:


\end{enumerate}
\end{multicols}
\end{document}
